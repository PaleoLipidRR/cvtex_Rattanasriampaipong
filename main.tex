%------------------------------------
% Dario Taraborelli
% Typesetting your academic CV in LaTeX
%
% URL: http://nitens.org/taraborelli/cvtex
% DISCLAIMER: This template is provided for free and without any guarantee 
% that it will correctly compile on your system if you have a non-standard  
% configuration.
% Some rights reserved: http://creativecommons.org/licenses/by-sa/3.0/
%------------------------------------

%!TEX TS-program = xelatex
%!TEX encoding = UTF-8 Unicode

\documentclass[10pt, letter]{article}
\usepackage{fontspec} 


% DOCUMENT LAYOUT
\usepackage{geometry} 
\geometry{letterpaper, textwidth=5.2in, textheight=8.5in, 
right=70pt,left=100pt,
top=1in,bottom=1in,
marginparsep=15pt, 
marginparwidth=1in}
\setlength\parindent{0in}

% FONTS
\usepackage[usenames,dvipsnames]{color}
\usepackage{xunicode}
\usepackage{xltxtra}
\defaultfontfeatures{Mapping=tex-text}
\setromanfont [Ligatures={Common}, Numbers={OldStyle}, Variant=01]{Linux Libertine O}

%% Additional packages
\usepackage{academicons}
\usepackage{fontawesome}

\usepackage{fancyhdr}
\renewcommand{\headrulewidth}{0pt}
\usepackage{lastpage}

\usepackage{natbib}
\usepackage{bibentry}
\newcommand{\myname}[1]{\textbf{Rattanasriampaipong, R.}}

\usepackage{etaremune}
\usepackage{enumitem}
\usepackage[yyyymmdd]{datetime}
\renewcommand{\dateseparator}{--}
\usepackage{xcolor}
\usepackage[inkscapeformat=png]{svg}

% ---- CUSTOM COMMANDS
\chardef\&="E050
\newcommand{\html}[1]{\href{#1}{\scriptsize\textsc{[html]}}}
\newcommand{\pdf}[1]{\href{#1}{\scriptsize\textsc{[pdf]}}}
\newcommand{\doi}[1]{\href{#1}{\scriptsize\textsc{[doi]}}}
% ---- MARGIN YEARS
\usepackage{marginnote}
\newcommand{\amper{}}{\chardef\amper="E0BD }
\newcommand{\years}[1]{\marginnote{\small #1}}
\renewcommand*{\raggedrightmarginnote}{}
\setlength{\marginparsep}{7pt}
\reversemarginpar
% ---- MARGIN TEXT
\newcommand{\margintext}[1]{\marginnote{\normalsize\textbf #1 |}}

% HEADINGS
\usepackage{sectsty} 
\usepackage[normalem]{ulem} 
\sectionfont{\mdseries\upshape\Large}
\subsectionfont{\mdseries\scshape\normalsize} 
\subsubsectionfont{\mdseries\upshape\large} 

% PDF SETUP
% ---- FILL IN HERE THE DOC TITLE AND AUTHOR
\usepackage[bookmarks, colorlinks, breaklinks, 
% ---- FILL IN HERE THE TITLE AND AUTHOR
	pdftitle={Ronnakrit Rattanasriampaipong - CV},
	% pdfauthor={My name},
	% pdfproducer={http://nitens.org/taraborelli/cvtex}
]{hyperref}  
\hypersetup{
  colorlinks,
  citecolor=Black,
  linkcolor=Black,
  urlcolor=Black}

% DOCUMENT
\begin{document}\thispagestyle{empty}
\textsc{\LARGE Ronnakrit Rattanasriampaipong} \hfill \textsc{Curriculum vitae}\\
\textsc{NOAA Climate \& Global Change Postdoctoral Fellow} \hfill (\textit{updated \today})\\

Gould-Simpson Building, Room 513A\\
Department of Geosciences, The University of Arizona \\
1040 E. 4th St., Tucson, AZ 85721
U.S.A.\\

\href{mailto:rrattan@ucar.edu}{rrattan@ucar.edu} | \href{mailto:ronnakritr@arizona.edu}{ronnakritr@arizona.edu} |
\href{https://www.researchgate.net/profile/Ronnakrit-Rattanasriampaipong?ev=hdr_xprf}{\aiResearchGate} 
\href{https://scholar.google.com/citations?user=AGcBWlAAAAAJ&hl=en}{\aiGoogleScholar} 
\href{https://www.scopus.com/authid/detail.uri?authorId=57826548100}{\aiScopus}
\href{https://www.webofscience.com/wos/author/record/HNT-0738-2023}{\includesvg[width=0.1in]{clarivate.svg}}
\href{https://orcid.org/0000-0002-1425-8737}{\aiOrcid} 
\href{https://github.com/PaleoLipidRR}{\faicon{github}}




%% RESEARCH INTERESTS
\bigskip
\margintext{Research Interests}
Proxy System Modeling, Mesozoic–Cenozoic Paleoceanography/Paleoclimatology, Archaeal Lipid Biomarker, Marine Ammonia Oxidizing Archaea, Archaeal Ecology and Evolution

%% APPOINTMENTS
\bigskip
\margintext{Appointment}
\textsc{The University of Arizona} \hfill Tucson, AZ \\
2023–present \quad NOAA Climate \& Global Change Postdoctoral Fellow \\
\textit{Mentor: Jessica Tierney}

%% EDUCATION
\bigskip
\margintext{Education}
\textsc{Texas A\&M University} \hfill College Station, TX \\
\textbf{2023} \quad Ph.D. in Oceanography (Paleoceanography/Organic Geochemistry) \\
\textit{Dissertation Committee Chair: Yige Zhang \\
Committee Members: Ethan Grossman, Robert Korty, Jason Sylvan \\}

\textsc{Chulalongkorn University} \hfill Bangkok, THA \\
\textbf{2016} \quad M.S. in Petroleum Geoscience (International Program)\\
\textit{Advisor: John K. Warren} \\

\textsc{Chulalongkorn University} \hfill Bangkok, THA \\
\textbf{2013} \quad B.S. in Geology \\
\textit{Advisor: Kruawun Jankaew}

\pagestyle{fancy}

%% Footer formatting
\fancyfoot{} % clear all footer fields
\fancyfoot[R]{Rattanasriampaipong, R. | 
\thepage\ of \pageref{LastPage}
}

%% AWARDS, HONORS
\bigskip
\margintext{Awards}
\textsc{Fellowships and Scholarships}
\begin{itemize}[leftmargin=*]
    \item[] \emph{NOAA Climate \& Global Change Postdoctoral Fellowship}, The Cooperative Programs for the Advancement of Earth System Science (CPAESS) at the University Corporation for Atmospheric Research (UCAR), 2023–2025
    \item[] \emph{Texas A\&M Dissertation Fellowship}, Texas A\&M University (TAMU), 2023–2024
    \item[] \emph{Texas Sea Grant's Grants-In-Aid Graduate Research Grant}, Texas Sea Grant, 2022–2024
    \item[] \emph{Oceanography Graduate Scholarship}, Department of Oceanography, TAMU, 2018–2024
    \item[] \emph{Schlanger Ocean Drilling Fellowship}, U.S. Science Support Program (USSSP) associated with the International Ocean Discovery Program (IODP), 2021–2022
    \item[] \emph{Fulbright Thai Graduate Scholarship}, Thailand-U.S. Education Foundation (TUSEF), Fulbright Thailand, 2018–2020
    \item[] \emph{Chevron Thailand Graduate Scholarship}, Chevron Thailand, 2015–2016
    \item[] \emph{Chevron Undergraduate Scholarship}, Chevron Thailand, 2012–2013
    \item[] \emph{Mitsui Oil Exploration Company (MOECO) Undergraduate Scholarship}, MOECO, 2011–2013 

\end{itemize}

\textsc{Honors and Awards}
\begin{itemize}[leftmargin=*]
    \item[] \emph{Invited student speaker} to give a reception speech at the Annual AGU Student Travel Grant Luncheon, AGU Fall Meeting, Chicago, IL, December 12th, 2022.
    \item[] \emph{Invited student speaker} to give a reception speech at the Award Recognition Ceremony, Ministry of Foreign Affairs, Bangkok, June 6th, 2018.
    \item[] \emph{Champion of The Hitachi Trophy 2013 Senior Project Competition}, ``won first place in Physical Science of the annual scientific pitching competition for all senior year students in the Faculty of Science," Chulalongkorn University, April 2013.
\end{itemize}

%% PUBLICATIONS
\bigskip
\margintext{Publications} \textsc{Manuscript in preparation}
\begin{enumerate}[noitemsep,nolistsep]
\nobibliography{references}
\item[4.] \bibentry{Rattanasriampaipong_MOB_inprep}
\end{enumerate}

\bigskip
\textsc{Published refereed journals}
\begin{etaremune}
\item \bibentry{Judd2022TheData}
\item \bibentry{Rattanasriampaipong2022}
\item \bibentry{Rattanasriampaipong16}
\end{etaremune}


%% PRESENTATIONS
\bigskip
\margintext{Presentations} 
\textsc{Academic \& Scientific Conferences}
\begin{etaremune}
\item {\myname{rattanasriampaipong}}, Zhang, Y. G., Alo, O., Liu, X. L., Zhang, Y., Kim, B., Marcantonio, F., and Bassinot, F. (2023, December). \textit{Bacterial tetraether lipids as a proxy for ocean (de)oxygenation}, Abstract (PP33B-04) presented at the 2023 AGU Fall Meeting, Moscone Center, San Francisco, CA, USA.

\item {\myname{rattanasriampaipong}}, Zhang, Y. G., Pearson, A., Hedlund, B., and Zhang, S (2022, December). \textit{Archaeal lipids suggest ecological shifts of marine ammonia-oxidizing archaea in greenhouse worlds}, Abstract (PP13C-04) presented at the 2022 AGU Fall Meeting, McCormick Place Convention Center, Chicago, IL, USA.

\item {\myname{rattanasriampaipong}}, Zhang, Y. G., Pearson, A., Hedlund, B., and Zhang, S (2022, April). \textit{Tracing ecology and evolution of marine ammonia-oxidizing archaea using archaeal lipid biomarkers}. Oral presentation for the departmental seminar at the Department of Oceanography, Texas A\&M University, TX, USA.

\item {\myname{rattanasriampaipong}}, Zhang, Y. G., Pearson, A., and Hedlund, B. (2021, December). \textit{Beyond TEX86: GDGTs Inform Marine Archaeal Community Ecology and Evolution}. Oral presentation (hybrid) at the 2021 AGU Fall Meeting, Ernest N. Morial Convention Center, New Orleans, LA, USA.

\item {\myname{rattanasriampaipong}}, Zhang, Y. G., Pearson, A., Hedlund, B., and Zhang, S. (2021, September). \textit{Beyond TEX86: GDGT Distributions Inform Archaeal Ecology}. Lightning (3-minute) oral presentation at the 2021 PhanTASTIC Workshop.

\item {\myname{rattanasriampaipong}} (2021, April). \textit{Closing the gaps of Cenozoic sea surface temperature history using tetraether archaeal lipid biomarkers}. Oral presentation for the departmental seminar at the Department of Oceanography, Texas A\&M University, TX, USA.

\item {\myname{rattanasriampaipong}} and Zhang, Y. G. (2019, December). Towards complete global sea surface temperature reconstructions over the Cenozoic Era. Poster presentation at the 2019 AGU Fall Meeting, Moscone Center, San Francisco, CA, USA.

\item {\myname{rattanasriampaipong}} (2013, April). \textit{Quantification of Tsunami Magnitude from Sedimentation Modeling of Re-occurring Indian Ocean Tsunamiites at Phra Thong Island, Phang Nga, Thailand}. Oral Presentation at the annual Hitachi Senior Project Competition, Faculty of Science, Chulalongkorn University, Thailand.
\end{etaremune}

\bigskip
\textsc{Industry}
\begin{etaremune}
\item {\myname{rattanasriampaipong}},  Marksamer, A., and Kantatong, P. (2018, April). \textit{A new approach for pore pressure prediction using neutron-density log separation}. Oral presentation at the 2019 Sub-Surface Technical Forum, Chevron Thailand Headquarter, Bangkok, Thailand.

\item {\myname{rattanasriampaipong}}, Paiboon, P., and Thatmali, P. (2015, February). \textit{Pattani Basin Regional Pore Pressure Study}. Oral Presentation at the annual meeting of Chevron Thailand Reservoir Management Forum, Swissotel Le Concorde, Bangkok, Thailand.

\item {\myname{rattanasriampaipong}}, Kananithikorn, N., Keawmoon, N., and Prasongtham, P. (2014, June). \textit{Causes of ballooning and lost circulation in Erawan and Satun fields}. Poster session presented at the bi-annual meeting of Chevron Reservoir Management Forum, The Woodlands Waterway Marriott Hotel, Houston, TX, USA
\item {\myname{rattanasriampaipong}}, Kananithikorn, N., Keawmoon, N., and Prasongtham, P. (2014, April). 
\textit{Causes of ballooning and lost circulation in Erawan and Satun fields}. Oral Presentation at the annual meeting of Chevron Thailand Reservoir Management Forum, Renaissance Hotel, Bangkok, Thailand.

\end{etaremune}

%% TALKS & LECTURES
\bigskip
\margintext{Invited Talks \& Lectures}
\textsc{Conducted in English}
\begin{etaremune}

\item \textit{Archaeal lipids trace ecology and evolution of marine ammonia-oxidizing archaea}, the geobiology session Where Rock Meets Life: Geobiology of Modern and Ancient Aquatic Ecosystems at the 50th SACNAS NDiSTEM, Portland Convention Center, Portland, OR, \textit{October 28, 2023. Talk.}

\item \textit{Marine AOA ecology shifts with Earth’s climate} (\href{https://tamucs-my.sharepoint.com/:b:/g/personal/rrattan_tamu_edu/EZQsnDcppXpAkxq9H8E_L1YBK_Jnd2zL1lPQwsRLIZZ_xw?e=ilpAa1}{slides}), The 2nd International GDGT Workshop, ETH Zurich, \textit{September 8, 2023. Talk.}

\item \textit{Archaeal lipids reveal distinct AOA ecology in past warm oceans} (\href{https://tamucs-my.sharepoint.com/:b:/g/personal/rrattan_tamu_edu/ET9FxsRS79dBv4giDLNmCKIB0X_tnPspwI18D1e8uKEFLw?e=6On32v}{slides}), Modern Geobiology Lecture Series, Department of Ocean Science and Engineering, Southern University of Science and Technology (SUSTech), \textit{June 12, 2023. Talk.}

\item \textit{A suppression of deep-water clades of marine ammonia oxidizers in past warm oceans}, Biology and Paleo Environment (BPE) Fall 2022 Seminar Series, Lamon-Doherty Earth Observatory (LDEO), Columbia University, \textit{October 24th, 2022. Talk.}

\item \textit{Untapped potential of archaeal lipids beyond ocean temperature reconstructions} (\href{https://www.youtube.com/watch?v=95o7ogv_T9I}{recording}), The Pal(a)eo EaRly Career Seminar (Pal(a)eoPERCS), Virtual Seminar, \textit{October 11th, 2022. Talk.}

\item 2021–22 Schlanger Ocean Drilling Student Fellow Research Talks, 2022 Summer Meeting of the U.S. Advisory Committee for Scientific Ocean Drilling (USAC), American Museum of Natural History, NY, \textit{July 25th, 2022. Talk.}

\item \textit{The Anthropocene: Human Footprints on Planet Earth} (\href{https://docs.google.com/presentation/d/1XTTknlXZghoFSpxdI-8KNPqmwrOvbwUQ9sDqNNvJDf0/edit?usp=sharing}{slides}), Marine Science Department, Chulalongkorn University, one lecture for Global Biogeochemical Cycles, Course instructor: Dr. Chawalit ``Net" Chareonpong, \textit{December 8th, 2021. Lecture.}
\end{etaremune}

\textsc{Conducted in Thai}
\begin{etaremune}
\item \textit{Ammonia Oxidizers in Past Warm Oceans} (\href{https://tamucs-my.sharepoint.com/:b:/g/personal/rrattan_tamu_edu/ESdkQzE0k-xNufbKOW7oMxwBUtcPdXOOMauWxMsfEqFgkA?e=Xk2nDl}{slides}), Department of Earth Sciences, Kasetsart University, one lecture for Dynamic Biosphere, Course instructor: Dr. Chatchalerm ``Kendo" Ketwetsuriya. \textit{January 10th, 2023. Lecture.}

\item \textit{Inferred paleoecology of marine archaea from today's oceans} (\href{https://fb.watch/grZcLZQSbV/}{recording}), Department of Marine Technology, Burapha University, Virtual Marine Technology Colloquium \#62, \textit{October 28th, 2022. Talk.}

\item \textit{Life after resignation: `Fulbright' has so much to offer} (\href{https://www.canva.com/design/DAE4hfYUXJk/3E9PACt31gAkqAKfjnURkg/edit?utm_content=DAE4hfYUXJk&utm_campaign=designshare&utm_medium=link2&utm_source=sharebutton}{slides}), Department of Geology, Faculty of Science, Chulalongkorn University, Virtual seminar, \textit{February 18th, 2022. Talk.}

\item \textit{Fossil lipids: Thermometers for the Earth's climate history} (\href{https://www.facebook.com/watch/?v=405982837057621}{recording}), Department of Marine Technology, Burapha University, Virtual Marine Technology Colloquium \#26, \textit{September 24th, 2020. Talk.}

\end{etaremune}

%% Teaching
\bigskip
\margintext{Teaching}
\textsc{Department of Oceanography, Texas A\&M University} \hfill College Station, TX \\
\textbf{Undergraduate Classes} \textit{(Class Size/Instructor Rating)} \\
\footnotesize 
Rating of \textit{``the instructor fostered an effective learning environment."}\\
Scale: 1 (Strongly Disagree) to 5 (Strongly Agree) \\

\normalsize
\textbf{\textsc{Graduate Assistant Lecturer}} \\
The Blue Planet - Our Oceans (OCNG 251) \\
\textit{Spring 2021 (116/4.78), Summer 2021 (36/4.86), Summer 2022 (48/4.64), Fall 2022 (194/4.41)}

\bigskip
\textbf{\textsc{Graduate Assistant - Teaching}} \\
The Blue Planet - Our Oceans (OCNG 251) \\
\textit{Spring 2023 (300/3.72)} \\
Data Analysis Methods in Geosciences (GEOS 470) \\
\textit{Spring 2023 (30/5.00)}

%% MENTORING
\bigskip
\margintext{Mentoring}
\textsc{Department of Oceanography, Texas A\&M University} \hfill College Station, TX \\
\textsc{\textbf{Zhang Lab Mentor}}\\
\footnotesize 
{\color{gray} Responsibilities: Trained on-campus and visiting undergraduate students on sample preparation (marine muds from ocean drilling programs) and lab procedures for archaeal lipid biomarker (specifically GDGTs) LC-MS analysis. Hands-on trainings are included, but not limited to: (1) freeze-drying samples, (2) homogenizing samples for total lipid extract (TLE) extraction using an Accelerated Solvent Extractor (ASE), (3) setting ACE methods, (4) purifying samples for LC-MS analysis (cellulose filtering and silica-gel column chromatography)}

\normalsize
\bigskip
Undergraduate Student Mentees: \textit{Connor Wood (TAMU | Fall 2022–Spring 2023), Roy Jui-Yu Huang (Emory University | Summer 2022), Ray Tarpey (TAMU | Spring 2022), Natalie York (TAMU | Fall 2021)} \\

\textsc{Chevron Thailand Exploration and Production Limited} \hfill Bangkok, THA \\
\textsc{\textbf{Geology Mentor}} | \textsc{Accelerated Earth Scientist Orientation Program (AESOP)}, 2014–15 \\
\footnotesize 
{\color{gray} Responsibilities: Trained two new-hired geologists on geology-related works for targeting hydrocarbon in Gomin D drilling project such as wireline logging interpretation, stratigraphic correlation, pore pressure prediction, and basic of well design, to achieve safe-and-effective drilling operations}

\normalsize
\bigskip
%% Leadership
\margintext{Leaderships}
\textsc{\textbf{Co-founder and Moderator}} | \textsc{Thai Earth and Planetary Scientists in North America} \\
Hosted and moderated semi-monthly discussions about earth and space sciences research for Thai students in the US who study earth and planetary sciences, May 2020–December 2020

\bigskip
\textsc{\textbf{Student President}} | \textsc{Undergraduate Geology Student Union}, 2012–13 \\
Department of Geology, Faculty of Science, Chulalongkorn University

%% Services
\bigskip
\margintext{Services}
\textsc{\textbf{Reviewer of manuscripts}} (9 reviews for 7 publications)\\submitted to the following refereed journals: \\
\textit{Frontiers in Marine Science (n=1), Geophysical Research Letters (n=2), Nature Geoscience (n=2), Organic Geochemistry (n=1), Paleoceanography and Paleoclimatology (n=2), Science Advances (n=1)} \\

\textsc{\textbf{Co-convener and co-chair}} of the session ``Past Climates and Environments of Southeast Asia and the Indo-Pacific" (PP42C \& PP45E) for the Paleoclimatology and Paleoceanography (PP) section at the 2022 AGU Fall Meeting. This is an inaugural dedicated PP session that highlights paleo-related works in Southeast Asia and the Indo-Pacific regions at AGU Fall Meeting. \\

\textsc{\textbf{Co-author}} of the open letter from Early Career Researchers (ECRs) to NSF ``On the Critical Importance of the U.S.-led Scientific Ocean Drilling" to express the concern regarding the future of NSF funding of U.S. scientific ocean drilling (SciOD). The letter was co-signed by 208 ECRs from 17 countries, representing 98 different institutions. \textit{July 8th, 2022.} \\

\textsc{\textbf{Website coordinator}} of Geochemistry and Paleoceanography Research Lab (Zhang Lab), Department of Oceanography, Texas A\&M University (TAMU), \textit{2018–23} \\

\textsc{\textbf{Graduate Student Senator}}, A representative for Oceanography Graduate Council at Graduate Student and Professional Government General Assembly, TAMU, \textit{2018–2022} \\
\begin{itemize}[noitemsep,nolistsep,leftmargin=*]
    \item Coauthored GPSG.R.53.05 Resolution ``Decarbonization and Fossil Fuel Divestment at Texas A\&M University." The GPSG senate passed the resolution on \textit{April 21st, 2020.}
\end{itemize}

\bigskip
\textsc{\textbf{Treasurer}}, Fulbright Students’ Association, TAMU, \textit{May 2019–May 2020} \\
\textsc{\textbf{Editor/Contributor}}, Climate Aware organization, \textit{January 2019–2020} \\
\textsc{\textbf{Tour Docent}}, R/V Sally Ride, AGU Fall Meeting 2019, San Francisco, CA, \textit{December 11th, 2019}\\
\textsc{\textbf{Board Member}}, FSA, TAMU, \textit{December 2018–April 2019}\\
\textsc{\textbf{Earth Scientist Knowledge Sharing Session Coordinator}}, Chevron Thailand, \textit{2017–18}


%% OUTREACH
\bigskip
\margintext{Outreach}
\textsc{Press and Media}\\
\textsc{Interviewed Articles in English}
\begin{etaremune}
\item \textit{Brave the World} (\href{https://issuu.com/fulbrightthailand/docs/storeis_fo_fulbrighters_impact}{online article}), Fulbright Thailand, Stories of Fulbrighters’ Impact—70th Anniversary of the Fulbright Program in Thailand, \textit{December 24th, 2021.}
\end{etaremune}

\textsc{Interviewed Articles in Thai}
\begin{etaremune}
\item \textit{Talk with Fulbrighter: Ronnakrit Rattanasriampaipong} (\href{https://www.fulbrightthai.org/fulbright-stories/khuykab-fulbrighter-rnkrsdi-ratnsrii-amaiphphngs}{online article}), Fulbright Thailand, \textit{May 18th, 2022.}
\item \textit{Recommended Field of Study: Paleoclimatology/Paleoceanography} (\href{https://www.fulbrightthai.org/knowledge-sharing/saakhaa-yaakaenanam-1-paleoclimatology-paleoceanography}{online article}), Fulbright Thailand, \textit{December 8th, 2021.}
\item \textit{Fulbright in My View} (\href{https://anyflip.com/afeb/buat/?fbclid=IwAR08dv9XF7UCimtM9SjwICdCSwSQfuTHNtPbbaepT4WZlTcNRXB9eNtUsi8}{online article}), Fulbright Thailand, Fulbright Experiences, \textit{October 6th, 2021.}
\item \textit{A journey without a map} (\href{https://www.facebook.com/ThaiScienceScholars/photos/a.114754726700839/170670231109288/?type=3}{online article}), The Science Scholars Facebook page, \textit{June 16th, 2020.}
\end{etaremune}

\textsc{Panel Discussions in Thai}
\begin{etaremune}
\item \textit{Fulbright Thai Graduate Scholarship (TGS): More Than Just a Scholarship} (\href{https://fb.watch/iLwkVkeNjJ/}{recording}), Fulbright Thailand, Virtual event, \textit{February 17th, 2023.}
\item \textit{Meet the Fulbrighter Series: Study Pure \& Applied Science in the U.S.} (\href{https://www.facebook.com/watch/?v=192470062421053}{recording}), Fulbright Thailand, Virtual event, \textit{March 31st, 2021.}
\item \textit{Meet the Fulbrighter Series: Doctoral Studies in the U.S.} (\href{https://www.facebook.com/watch/?v=1018096972055863}{recording}), Fulbright Thailand, Virtual event,\textit{ March 5th, 2021.} 
\end{etaremune}

\bigskip
%% Professional Experience
\margintext{Professional Experience}
\textsc{Chevron Thailand Exploration and Production Limited} \hfill Bangkok, THA \\
\textsc{\textbf{Development Geologist}} | \textsc{Satun-Funan Asset Team}, 2016–18 \\
\textsc{\textbf{Horizons Earth Scientist}} | \textsc{AESOP}, 2013–15

\bigskip
Highlighted Responsibilities:
\begin{itemize}[leftmargin=*]
    \item Identify hydrocarbon potentials and conduct detailed geological assessments within the extensional basin (mainly fluvio-deltaic depositional environments) based only multiple data sets including wireline logs, 3D seismic, and historical production data
    \item Manipulate geological data from over 5,000 drilled wells and establish meaningful data visualizations – such as detailed pore pressure analysis and geo-statistical reserves booking
    \item Integrate cross-disciplinary data including geological, petroleum engineering, and drilling engineering data to optimize hydrocarbon exploration and production
    \item Re-evaluate and discriminate regional pore pressure and temperature regime to enhance the accuracy of predrilled reserves prediction for all undeveloped projects in the Gulf of Thailand
    \item Developed a new method for reservoir pressure evaluation using wireline logging and gas show
Student Intern, Reservoir Management Team, two consecutive summers, April–May 2012/13
    \item Wireline logging interpretation (triple combo suites) and stratigraphic correlation (mainly fluvio-deltaic depositional environments)
    \item Pore pressure profile prediction and oil and gas reserves estimation for YUWA platform
    \item Re-targeting geological targets for nine future platforms in Dara geologic trend
\end{itemize}

\textsc{\textbf{Undergraduate Student Intern}} | \textsc{Reservoir Management Team}, Summer 2012/23

\bigskip
Highlighted Responsibilities:
\begin{itemize}[leftmargin=*]
    \item Well logging interpretation (triple combo suites) and stratigraphic correlation (mainly fluvio-deltaic depositional environments)
    \item Pore pressure profile prediction and oil and gas reserves estimation for YUWA platform
    \item Re-targeting geological targets for nine future platforms in Dara geologic trend


\end{itemize}

\bigskip
%% Previous Research Experience
\margintext{Previous Research Experience}
\textsc{Department of Oceanography, Texas A\&M University} \hfill College Station, TX \\
\textsc{\textbf{Doctoral Dissertation Research}}, 2018–23 | \textit{``Beyond TEX86: Evaluating Archaeal Evolution Coupled with Oceans and Climate Changes using Tetraether Lipids"} | Advisor: Yige Zhang \\

\textbf{\textsc{Chapter 1.}} Towards continuous sea surface temperature records inferred from archaeal lipids over the past 66 million years
\begin{itemize}[leftmargin=*]
    \item Identified spatio-temporal gaps in publicly available Cenozoic TEX86 records
    \item Collected core samples at Gulf Coast Repository (GCR) (College Station, TX) for lipid biomarker analysis (multiple visits)
    \item Conducted analytical work for coarse-resolution SST reconstructions from ten ODP/IODP sites
    \item Revisiting temperature calibrations with the new insight into the evolution of marine AOA paleoecology
\end{itemize}   

\bigskip
\textbf{\textsc{Chapter 2.}} Archaeal lipids trace ecology and evolution of marine ammonia-oxidizing archaea
\begin{itemize}[leftmargin=*]
    \item Compiled and quality controlled an extensive dataset (n = $\approx$4,000 entries from 79 publications) of isoprenoidal glycerol dialkyl glycerol tetraethers (GDGTs) derived from different archives
    \item Identified diagnostic distribution patterns of GDGT assemblages that trace marine archaeal ecology and evolution
    \item Utilized modern statistical analyses and unsupervised clustering algorithms
\end{itemize}    


\textbf{\textsc{Chapter 3.}} Investigating changes of oxygen in past oceans through the lens of lipid biomarkers
\begin{itemize}[leftmargin=*]
    \item Investigating the potential of overly-branched (OB-), branched (B-), and sparsely-branched (SB-) GDGTs as a proxy for oxygen deoxygenation events in past oceans
    \item Integrating available archaeal lipid data with gridded World Ocean Atlas (WOA) datasets, including temperature and oxygen, to extract modern-day distributions of the known range of glycerol ether lipids in several settings, including Black Sea, Arabian Sea, Eastern Equatorial Pacific (EEP)
\end{itemize}  

\bigskip
\textsc{Department of Geology, Chulalongkorn University} \hfill Bangkok, THA \\
\textsc{\textbf{Masters Thesis Research}}, 2016 | \textit{``Potential Sources of Mercury in Southern Pattani Basin, the Gulf of Thailand"} | Advisor: John Warren 

\begin{itemize}[leftmargin=*]
    \item Conducted the first geological assessment of mercury-contaminated produced hydrocarbon in the Gulf of Thailand with multi-disciplinary data including C-O isotopic analysis in siliciclastic cuttings
    \item Established and applied C-O isotopic analysis to siliciclastic cuttings
    \item Established correlation of Hg-CO2 for multiple scales; from reservoirs to platforms scale
\end{itemize}  

\bigskip
\textsc{Department of Geology, Chulalongkorn University} \hfill Bangkok, THA \\
\textsc{\textbf{Undergraduate Senior Year Research}}, 2011–13 | \textit{``Quantification of Tsunami Magnitude from Sedimentation Modeling of Re-occurring Indian Ocean Tsunamiites at Phra Thong Island, Phang Nga, Thailand"} | Advisor: Kruawun Jankaew
\begin{itemize}[leftmargin=*]
    \item Tsunami sample preparation for grain size analysis, Earth Observatory of Singapore (EOS), Nanyang Technological University (NTU), Singapore, May 2013
    \item Fieldwork and sample collection of tsunami sediments, Phrathong Island, Phang-nga, Thailand, June 2011 - July 2013 (multiple field campaigns)
    \item Collaborated with research teams from EOS-NTU and Stockholms Universitet
\end{itemize}  

\bigskip
%% Cruise/Field Exprience
\margintext{Cruise/Field Experience}
\textsc{Department of Oceanography, Texas A\&M University} \hfill College Station, TX \\
\textsc{\textbf{Research Staff}}, November 3rd, 2018 
\begin{itemize}[leftmargin=*]
    \item Collected seawater samples with Niskin bottles in Galveston Bay
    \item Measured seawater properties (Conductivity, Temperature, Depth) using handheld CTD 
    \item Observed other oceanographic sampling operations: seafloor grab sampling, water turbidity using Secchi disk, seawater pumping for biological and trace metal analyses
\end{itemize}  

\bigskip
\textsc{Chevron Thailand Exploration and Production Limited} \hfill Bangkok, THA \\
\textsc{\textbf{Shore-based Operational Geologist}}, July–December 2017
\begin{itemize}[leftmargin=*]
    \item Two infill projects in Jakrawan Gas Field: 20 wells drilled with 90\% accuracy between pre-drill and post-drill hydrocarbon reserves estimations
    \item Conducted real-time stratigraphic correlation (wireline triple combo) and re-evaluated pore pressure profile during drilling operations
    \item Evaluated economic justifications on a well-by-well basis to optimize project costs and maximize hydrocarbon resources
\end{itemize} 

\bigskip
\textsc{Chevron Thailand Exploration and Production Limited} \hfill Bangkok, THA \\
\textsc{\textbf{Wellsite Geologist Trainee }}, November 2nd–13th, 2014
\begin{itemize}[leftmargin=*]
    \item Observed and learned real-time drilling operations at Moragot E platform, Moragot gas field, the Gulf of Thailand
    \item Key learning activities: Formation pressure acquisition, Formation Integrity Test (FIT), Rig-floor operations 
\end{itemize}

\bigskip
%% Technical Reports
\margintext{Technical Reports}
\textsc{Academia}
\begin{itemize}[leftmargin=*]
    \item[] {\myname{rattanasriampaipong}} and Jankaew, K. (2012) \textit{Quantification of Tsunami Magnitude from Sedimentation Modeling of Re-occurring Indian Ocean Tsunamiites at Phra Thong Island, Phang Nga, Thailand}. Undergraduate Senior Year Researh Thesis.
\end{itemize}   

\bigskip
\textsc{Industry} \\
\textsc{\textbf{Recommended Location, Logging and Coring Program (RLLCP) Geological Reports}}, 2013–18
\begin{itemize}[leftmargin=*]
    \item Provided a geological overview of the targeted intervals for 4 drilling projects including Satun N, Moragot F, Jakrawan G, and Jakrawan D
    \item Assessed geological risks and evaluated hydrocarbon potentials of the project
    \item Designed wireline logging program including formation testing
\end{itemize} 

\bigskip
%% Workshops
\margintext{Workshops}
\textsc{Professional Development}

\begin{itemize}[leftmargin=*]
    \item[] Paleoclimate Data Assimilation: Challenges, Innovations, \& Opportunities, The University of Arizona, Tucson, AZ, \textit{October 18th–21st, 2022.}
    \item[] PaleoCAMP 2022 summer school, the inaugural Heising Simons Foundation-sponsored paleoclimate short course, Sierra Nevada Aquatic Research Laboratory (SNARL), Mammoth Lakes, CA, \textit{July 11th–22nd, 2022. (25 attendees from 132 applicants)}
    \item[] Science Mission Requirements (SMR) Workshop for a Globally-Ranging Riserless U.S. Drilling Vessel, NSF-sponsored USSSP-IODP workshop, Chicago, IL, \textit{May 17th–18th, 2022}
    \item[] LinkedEarth PaleoHackathon 2, Virtual Python Workshop, \textit{October 28th–29th, 2021 }
\end{itemize} 

\bigskip
\textsc{G.R.A.D. Aggies Professional Development Program} \hfill Texas A\&M University
\begin{itemize}[leftmargin=*]
    \item[] NSF: The Agency, Proposal Preparation \& Review, \textit{February 19th, 2020}
    \item[] Open Educational Resources Workshop, \textit{September 13th, 2019}
    \item[] Creating a Life of Balance \& Wellness, \textit{September 10th, 2019}
\end{itemize} 

\bigskip
\textsc{Foreign Fulbright Student Program}
\begin{itemize}[leftmargin=*]
    \item[] Innovations in Civic Engagement: Harnessing Data for the Public Good, Philadelphia, PA, \textit{March 7th–10th, 2019}
    \item[] Pre-Academic Orientation Program, Ohio University, Athens, OH, \textit{July 28th–August 18th, 2018}
\end{itemize} 

\bigskip
\textsc{Chevron In-house Training by Energy Company Technology}
\begin{itemize}[leftmargin=*]
    \item[] Reservoir Geophysics, \textit{November 21st–25th, 2017}
    \item[] Applied Concepts of Structural Geology, \textit{June 12th–16th, 2017}
    \item[] Stratigraphic Analysis of Shallow Marine and Fluvial Reservoirs, \textit{May 29th–June 2nd, 2017} 
    \item[] Basic Reservoir Engineering, \textit{December 15th–19th, 2014}
    \item[] Formation Evaluation (Fundamentals), \textit{October 6th–9th, 2014}
    \item[] Applied Stratigraphic Concepts, \textit{August 4th–8th, 2014}
    \item[] Applied Subsurface Geological Mapping, \textit{June 23rd–27th, 2014}
    \item[] Applied Petroleum Geochemistry, \textit{April 21st–24th, 2014}
\end{itemize} 

\bigskip
\textsc{Chevron In-house Training by Nautilus}
\begin{itemize}[leftmargin=*]
    \item[] Interpretation of 3D Seismic Data, \textit{November 14th–17th, 2016}
\end{itemize} 

\bigskip
%% Professional Societies
\margintext{Professional Societies}
\textbf{Student Member}, American Association for the Advancement of Science (AAAS), 2019–Recent \\
\textbf{Student Member}, American Geophysical Union (AGU), 2018 - Recent \\
\textbf{Student Member}, American Association of Petroleum Geologists (AAPG), 2013–Recent \\
\textbf{Secretary}, Thailand Society of Exploration Geophysicists (TSEG), 2015–16 \\
\textbf{Student Member}, Society of Petroleum Engineers (SPE), 2010–13

\bigskip
%% Skills
\margintext{Skills}
\textbf{Documentation and artworks:} MS Office Suites, LaTeX (Overleaf), Adobe Illustrator \\
\textbf{Data Analytics and Visualizations:} Python, MATLAB, ArcGIS, Tableau and Spotfire

\bigskip
%% Languages
\margintext{Languages}
Thai (Native), English (Full professional proficiency)

\bigskip
%% Personal Interests
\margintext{Personal Interests} 
Ultra-trail running, Marathon, Non-fiction books, Science communication, Infographics, Data visualization

\bibliographystyle{apalike}
\end{document}